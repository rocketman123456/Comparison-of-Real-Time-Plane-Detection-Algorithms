\documentclass[main.tex]{subfiles}
\begin{document}
\chapter{Introduction}\label{chap:Introduction}

Man-made environments usually contain planar structures to a large extent.
They are a central component in numerous use cases in the fields of Augmented and Virtual Reality, as well as robotics.
% FIXME "sprung zu gross!" 

\section{Real-Time Plane Detection}

\paragraph*{introduction}
\begin{enumerate}
    \item aktueller stand: gute und schnelle ebenenfindung wird oft gebraucht, gibt es auch schon/ist möglich
    \item problem: oft sind die speziellen sensoren sehr kostenspielig
    \item Daher die frage: (wie gut) ist das ganze auf bezahlbarer hardware möglich?
    \item Nötig, um die Frage zu beantworten: 
        \begin{itemize}
            \item Welche Kamera(s)? % FIXME auf die realsense beschränken
            \item Welcher algorithmus?
            \item Was heisst "real-time" überhaupt?
        \end{itemize} 
        \item Problem an letzterem: nicht möglich einen einheitlich besten algorithmus auszuwählen, da \dots{}  % FIXME hier nur kurz anreissen
    \item Lösung: wir wählen algorithmen aus und vergleichen diese einheitlich um die frage aus 3. zu beantworten
\end{enumerate}

\section{Intel RealSense}
% FIXME auch mit in die INTRO
\begin{itemize}
    \item Wir müssen zuerst sensoren auswählen, mit denen wir die umgebung aufnehmen
    \item Da, wie vorher angesprochen, der preis des sensors oft ein problem ist, wählen wir eine relativ billige (im vergleich)
    \item die intel sensoren sind vergleichsweise bezahlbar. 
    \item genauer gesagt nutzen wir \dots{}
    \item Zu den sensoren wird eine kostenfreie software bereit gestellt
    \item Über diese software lassen sich die kameras ansteuern. dazu ist in dieser software ein slam algorithmus namens rtabmap implementiert
    \item mit rtabmap können wir den strom aus rohdaten zu einer bestehenden karte verarbeiten, was uns ermöglicht ebenen der kompletten umgebung zu finden anstatt nur von dem aktuellen blickwinkel
\end{itemize}



% \paragraph*{What is Real-Time Plane Detection (RTPD)}
% Some use cases are subject to temporal restriction of some degree. If the temporal constraint is minimal or a task must be executed as quickly as possible, the term \textit{real-time} is often used. 
% In the context of plane detection, real-time performance is often necessary when the subject is in motion. 
% A possible scenario could involve an autonomous robot moving through a building. Here, the process of detecting planar surfaces in the vicinity is necessary as, otherwise, the robot could crash into a nearby wall or Door.

% \paragraph*{Problem of RTPD}
% T\begin{itemize}
%     \item die technik existiert schon
%     \item RTPD braucht spezielle sensoren (/kameras)
%     \item diese kameras sind leider oft kostenspielig, was sich nicht jeder leisten kann (vgl leico oder HL)
% \end{itemize}

% \paragraph*{wie gehen wir das problem an}
% \begin{itemize}
%     \item daher die frage ob das ganze nicht auf auf bezahlbarer hardware geht
% \end{itemize}
% Um ebenenfindung benutzerfreundlicher und allgemein inklusiver zu machen, wird in dieser
% arbeit daher die möglichkeit der ebenenfindung auf off-the-shelf hardware evaluiert.

% Dies wird angegangen, indem verschiedene PDA in einer strukturierten literaturrecherche verglichen werden


% \section{Structure of this work}

\end{document}