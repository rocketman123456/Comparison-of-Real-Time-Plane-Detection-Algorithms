\documentclass[main.tex]{subfiles}
\begin{document}
\chapter{Introduction}\label{chap:Introduction}

Man-made environments usually contain planar structures to a large extent.
They are a central component in numerous use cases in the fields of Augmented and Virtual Reality, as well as robotics.


\section{Real-Time Plane Detection}
\paragraph*{What is Real-Time Plane Detection (RTPD)}
Some use cases are subject to temporal restriction of some degree. If the temporal constraint is minimal or a task must be executed as quickly as possible, the term \textit{real-time} is often used. 
In the context of plane detection, real-time performance is often necessary when the subject is in motion. 
A possible scenario could involve an autonomous robot moving through a building. Here, the process of detecting planar surfaces in the vicinity is necessary as, otherwise, the robot could crash into a nearby wall or Door.

\paragraph*{Problem of RTPD}
T\begin{itemize}
    \item die technik existiert schon
    \item RTPD braucht spezielle sensoren (/kameras)
    \item diese kameras sind leider oft kostenspielig, was sich nicht jeder leisten kann (vgl leico oder HL)
\end{itemize}

\paragraph*{wie gehen wir das problem an}
\begin{itemize}
    \item daher die frage ob das ganze nicht auf auf bezahlbarer hardware geht
\end{itemize}
Um ebenenfindung benutzerfreundlicher und allgemein inklusiver zu machen, wird in dieser
arbeit daher die möglichkeit der ebenenfindung auf off-the-shelf hardware evaluiert.

Dies wird angegangen, indem verschiedene PDA in einer strukturierten literaturrecherche verglichen werden


\section{Structure of this work}


In the following chapter, the basics are explained. In the concept\ref{chap:Concept}, the problem is defined in more detail and a solution is proposed. 
The fourth chapter focuses on the implementation details of the previous chapter.
The proceding chapter deals with the comparison of plane detection algorithms and the evaluation thereof.
Finally, the results and future work are presented. 

\end{document}