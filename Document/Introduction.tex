\documentclass[main.tex]{subfiles}
\begin{document}
\chapter{Introduction}\label{chap:Introduction}

Planare Flächen sind in einer Vielzahl von Menschen gemachten Umgebungen zu finden. 
Sie sind ein zentraler Bestandteil in zahlreichen Anwendungsfällen aus den Bereichen der 
Augmented und Virtual Reality, sowie der Robotik.


\section{Real-Time Plane Detection}
\paragraph*{Was ist RTPD}

Es gibt anwendungsfälle, welche einer temporalen einschränkung unterliegen. Ist diese temporale einschränkung sehr klein, oder grob gesagt "so schnell wie möglich", 
wird auch oft von "echtzeit" gesprochen. Im kontext der ebenenfindung kommt oft ein echtzeit anspruch dazu, wenn der anwender sich in bewegung befindet.
Bewegt sich also ein autonomer roboter durch ein gebäude, ist es wichtig, dass schnell genug alle ebenen im sichtfeld gefunden werden, damit eine kollision mit der wand vermieden wird.


\paragraph*{problem an RTPD}

Echtzeit ebenenfindung ist bisher möglich, jedoch teils unter einer dicken paywall verborgen, siehe die
microsoft hololens. 
Ausserdem ist bisher kein einheitlicher vergleich angestellt worden, daher ist es nicht direkt möglich einen eindeutig besten algorithmen anhand einer literaturrecherche auszuwählen.


\paragraph*{wie gehen wir das problem an}
Um ebenenfindung benutzerfreundlicher und allgemein inklusiver zu machen, wird in dieser
arbeit daher die möglichkeit der ebenenfindung auf off-the-shelf hardware evaluiert.

Dies wird angegangen, indem verschiedene PDA in einer strukturierten literaturrecherche verglichen werden


\section{Structure of this work}

Im folgenden kapitel werden die grundlagen erklärt. Im concept wird das problem etwas genauer definiert,
sowie der lösungsansatz berichtet. 
Das darauf folgende kapitel fokussiert sich auf die realisierung des konzepts.
Im 5. kapitel werden die PDAs verglichen und evaluiert.
Zu guter letzt werden die ergebnisse und future work präsentiert. 

\end{document}