\documentclass[main.tex]{subfiles}
\begin{document}
\chapter{Introduction}
\label{chap:Introduction}

\textbf{\textcolor{red}{ist noch ein wenig scrambled. grundlegende toc:}}
\begin{enumerate}
    \item \textbf{Motivation}
    \begin{itemize}
        \item ebenen sind überall
        \item plane detection ist voll oft genutzt und voll cool!
        \item PD wird in diversen applikationen benutzt
        \item manche dieser applikationen sind zeitlich eingeschränkt -> realtime
        \item problem 1: sensor noise, grade indoors
        \item ansatz 1: sensor improvement
        \item problem 2: is' teuer, kann sich nicht jeder leisten
        \item ansatz 2: bezahlbare sensoren -> intel RS, bester algorithmus / wie gut geht das ganze
        \item problem 3: PDAs sind nicht vergleichbar
    \end{itemize}
    \item \textbf{Goals}
    \begin{itemize}
        \item plan: einheitlicher vergleich
        \item ergbnis: wie gut geht PD auf OTS hardware, welcher algo ist der beste
        \item einschänkungen auf indoor, ganze environments
    \end{itemize}
    \item \textbf{structure}
\end{enumerate}
Menschen gemachte Umgebungen bestehen aus einem Großteil aus planaren strukturen. Planare strukturen finden sich oft in den Grundbausteinen von Gebäuden wieder,
 ein Großteil der Innenausstattung hat flache Oberflächen, sogar das Lesen dieser Arbeit findet wahrscheinlich über ein planares Medium statt.
Durch den großen anteil an planaren strukturen in alltäglichen umgebungen, besonders indoor, gewinnt das automatische erkennen von diesen ebenen an relevanz:
Plane detection ist ein wichtiger bestandteil in vielen applikationen unter anderem in der robotik, der Virtual oder Augmented Reality (AR/VR). 

Viele dieser Systeme arbeiten dazu unter gewissen zeitlichen einschränkungen, was ebenso die maximal verfügbare berechnungszeit der plane detection beeinflusst.
Man spricht auch oft von real-time, wenn so eine applikation ihre berechnungen innerhalb einer festen zeitvorgabe ausführen muss.
Eine der haupt challenges der plane detection ist, dass die sensoren oft ein hohes mass an noise haben, was die berechnungen kompliziert, die präzision der 
algorithmen vermindert, oder sogar die berechnung ganz unmöglich macht. Das level der noise hängt stark von der benutzung der kamera ab. 
Die meisten AR/VR applikationen, sowie viele SLAMs werden in indoor umgebungen ausgeführt. Durch schnelle bewegung in räumlich engen umgebungen sinkt die 
präzision der aufgenommenen daten schnell.

Ein möglicher lösungsansatz ist natürlich, die sensorik zu verbessern, mit der die umgebung aufgenommen wird. Mit steigender sensor präzision steigt aber auch schnell der preis des sensors.
Beispielsweise kann die microsoft hololens plane detection ausführen. Dies passiert direkt intern als teil des sogenannten spatial mappings.   
Die berechnungszeit und präzision davon ist gut, was man für ehr als $3000$ Euro neupreis auch erwarten kann. ein weiteres beispiel wären sensoren wie der leica blk360. welcher zwar keine plane
detection durchführt, dafür liefert dieser aber sehr dichte und präzise punktwolken.
Auch wenn diese technik sehr präzise daten liefert, wird sie sich der durchschnittliche benutzer nicht leisten können.

Das wirft die Frage auf, wie gut plane detection sein kann, wenn man bezahlbare Sensoren benutzt. 
Mit insgesamt \textbf{\textcolor{red}{600?}} euro sind die T256 und D455 der intel realsense reihe ein beispiel für bezahlbare sensorik. 
Neben den sensoren ist natürlich auch der plane detection algorithmus selbst von zentraler bedeutung. Über jahrzehnte der forschung wurden viele algorithmen entworfen.
Das resultiert aber in dem nächsten problem; Die meisten algorithmen sind nicht objektiv mit einander vergleichbar. Das kann verschiedene Gründe haben, oft 
wurden algorithmen auf verschiedenen datensätzen getestet, es wurden verschiedene metriken benutzt, oder die Daten, auf denen die algorithmen arbeiten sind einfach grundlegend
nicht vergleichbar. 


\section{Goals}

Diese Arbeit beschäftigt sich mit einen einheitlichen Vergleich ausgewählter plane detection algorithmen.
Durch diesen Vergleich soll zu, einen die Frage bewertet werden, in welchem Ausmass präzise real-time plane detection auf 
bezahlbarer hardware wie der Intel RealSense möglich ist. Durch die beantwortung davon ergibt sich im zuge auch, welcher Algorithmus 
am besten geeignet ist.
Dazu sei gesagt, dass wir uns hierbei auf die plane detection in vollständigen 3D Umgebunden fokussieren. Ferner beschränken wir uns auf die plane detection in 
indoor umgebungen, die Daten werden also hauptsächlich im inneren von Gebäuden aufgenommen. 

\section{Structure}
Chapter 2 präsentiert nötiges Hintergrundwissen für diese Arbeit.

Im Konzept (chapter 3) wird der grundsätzliche Leitfaden dieser arbeit Festgelegt.

Chapter 4 detailliert die implementations details.

Der einheitliche Vergleich wird in Kapitel 5 stattfinden. Darüber hinaus werden die Ergebnisse ausgewertet.

Wir werden das Fazit ziehen und somit die Frage dieser arbeit im letzten kapitel (chapter 6) beantworten.
\end{document}
