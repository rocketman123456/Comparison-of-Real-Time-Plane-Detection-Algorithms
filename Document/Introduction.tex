\documentclass[main.tex]{subfiles}
\begin{document}
\chapter{Introduction}\label{chap:Introduction}

Man-made environments usually contain planar structures to a large extent. 
They are a central component in numerous use cases in the fields of Augmented and Virtual Reality (AR/VR), as well as robotics.
In these applications, it is common to record an environment with specific sensors, 
process the data separately, and finally try to detect planes within the data. 
Naturally, each of these individual steps has the potential for improvement. 
Moreover, an improvement of the sensor often results in a subsequent improvement of the remaining steps.
However, a problem arises: With an increasing demand for sensor quality, the cost of this technology 
increases as well. For instance, the starting price of the \textit{Leica BLK360} laser scanner starts at $19,000$ euro,
 and the price of a \textit{Microsoft HoloLens 2} begins at ${\sim}3,400$ euro. In the case of the \textit{HoloLens},
a plane detection process is already integrated. 
The problem remains that some hardware, while high quality, is not affordable for the average consumer.

Another important aspect is that many applications include some form of temporal component. For instance, 
an autonomous robot cannot have significant delays, since it is otherwise likely to crash into an obstacle.
Similarly, the user experience is dramatically decreased if the application in use takes a long time to process 
an update. In the context of plane detection, this could mean that the user looks in a new direction but has 
to wait until the update is processed and new planes are detected.

Naturally, with sensors that provide integrated plane detection, the additional temporal component leads to a higher price.

This raises the question as to what the limits of plane detection on "off-the-shelf" hardware are, especially including the 
temporal component. 

In the following section, we introduce the primary focus of this work: \textit{Real-Time Plane detection.}
\section{Real-Time Plane Detection}

\textit{Real-Time Plane detection} is a broad term for algorithms, systems, or entire applications, that 
perform the detection of planar regions under a certain temporal constraint. Depending on the underlying 
use case, this temporal constraint varies. For instance, an architect that needs a digital model of a basement has 
a smaller constraint than a digital VR chatroom like \textit{VRChat}\footnote{\href{https://hello.vrchat.com/}{,https://hello.vrchat.com/}}.


\begin{enumerate}
    \item aktueller stand: gute und schnelle ebenenfindung wird oft gebraucht, gibt es auch schon/ist möglich
    \item problem: oft sind die speziellen sensoren sehr kostenspielig
    \item Daher die frage: (wie gut) ist das ganze auf bezahlbarer hardware möglich?
    \item Nötig, um die Frage zu beantworten: 
        \begin{itemize}
            \item Welche Kamera(s)? % FIXME auf die realsense beschränken
            \item Welcher algorithmus?
            \item Was heisst "real-time" überhaupt?
        \end{itemize} 
        \item Problem an letzterem: nicht möglich einen einheitlich besten algorithmus auszuwählen, da \dots{}  % FIXME hier nur kurz anreissen
    \item Lösung: wir wählen algorithmen aus und vergleichen diese einheitlich um die frage aus 3. zu beantworten
\end{enumerate}

\section{Intel RealSense}
% FIXME auch mit in die INTRO
\begin{itemize}
    \item Wir müssen zuerst sensoren auswählen, mit denen wir die umgebung aufnehmen
    \item Da, wie vorher angesprochen, der preis des sensors oft ein problem ist, wählen wir eine relativ billige (im vergleich)
    \item die intel sensoren sind vergleichsweise bezahlbar. 
    \item genauer gesagt nutzen wir \dots{}
    \item Zu den sensoren wird eine kostenfreie software bereit gestellt
    \item Über diese software lassen sich die kameras ansteuern. dazu ist in dieser software ein slam algorithmus namens rtabmap implementiert
    \item mit rtabmap können wir den strom aus rohdaten zu einer bestehenden karte verarbeiten, was uns ermöglicht ebenen der kompletten umgebung zu finden anstatt nur von dem aktuellen blickwinkel
\end{itemize}



% \paragraph*{What is Real-Time Plane Detection (RTPD)}
% Some use cases are subject to temporal restriction of some degree. If the temporal constraint is minimal or a task must be executed as quickly as possible, the term \textit{real-time} is often used. 
% In the context of plane detection, real-time performance is often necessary when the subject is in motion. 
% A possible scenario could involve an autonomous robot moving through a building. Here, the process of detecting planar surfaces in the vicinity is necessary as, otherwise, the robot could crash into a nearby wall or Door.

% \paragraph*{Problem of RTPD}
% T\begin{itemize}
%     \item die technik existiert schon
%     \item RTPD braucht spezielle sensoren (/kameras)
%     \item diese kameras sind leider oft kostenspielig, was sich nicht jeder leisten kann (vgl leico oder HL)
% \end{itemize}

% \paragraph*{wie gehen wir das problem an}
% \begin{itemize}
%     \item daher die frage ob das ganze nicht auf auf bezahlbarer hardware geht
% \end{itemize}
% Um ebenenfindung benutzerfreundlicher und allgemein inklusiver zu machen, wird in dieser
% arbeit daher die möglichkeit der ebenenfindung auf off-the-shelf hardware evaluiert.

% Dies wird angegangen, indem verschiedene PDA in einer strukturierten literaturrecherche verglichen werden


% \section{Structure of this work}

\end{document}