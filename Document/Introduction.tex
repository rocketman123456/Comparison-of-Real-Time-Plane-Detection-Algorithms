\documentclass[main.tex]{subfiles}\begin{document}
\chapter{Introduction}
\label{chap:Introduction}

Man-made environments usually consist, to a large extent, of planar structures.
These planar structures, or planes, are often found in the basic geometry and alignment of buildings~\cite{Coughlan_Yuille_1999}, and much of the design of interior furniture has flat surfaces~\cite{Zhao_Zong_Cao_Chen_Liu_Xu_Xin_2020}.
Due to this large amount of planes in everyday environments, especially indoors, automatic detection of these planes is growing in relevance:
Plane detection is an essential component in applications of numerous fields, including robotics~\cite{Zhang_Wang_Xianyu_Ziwei_Wei_2019, Yunus2021ManhattanSLAMRP, Zhang_Zeng_Zha_2017}
and virtual or augmented reality (AR/VR)~\cite{Jurado_Jurado_Ortega_Feito_2021, sridhar2020instant}.% TODO , \textbf{\textcolor{red}{und XY [source]}}.

% https://apps.apple.com/us/app/ar-plan-3d-camera-to-plan/id1459846158 <- beispiel applikation

Many of these applications operate under specific time constraints. These constraints are often broadly referred to as \textit{real-time}.
The definition of \textit{real-time} usually derives from the frequency of new sensor updates~\cite{Davison_2003}.
Since the process of plane detection is often an integral part of these systems ~\cite{Wang_Bu_Zhang_Cheng_2022, Dai_Lund_Gao_2022, Kaess_2015}, these constraints also apply there.

\textit{Real-time plane detection} is already possible, though expensive hardware is often needed as a sensor's price increases 
with its precision.
For instance, AR devices like the \textit{Microsoft HoloLens 2} or imaging laser scanners like the \textit{Leica BLK360} produce very precise representations of the surrounding environment, and
in the former case, even provide the functionality to perform plane detection internally. Nevertheless, with starting prices of ${\sim}\$3.5k$ and ${\sim}\$19k$, respectively, these devices are likely
not affordable for the average consumer.

Through this lack of affordability of high-end sensors, the usage of \textit{off-the-shelf} hardware is gaining interest.
At a total of ${\sim}\$600$, the \textit{Intel RealSense} cameras T265 and D455 are an example of more affordable sensor technology (see Section~\ref{sec:bg-intel}).

In addition to the used sensors, selecting an appropriate plane detection algorithm is important as well.
Many algorithms have been designed over decades of research.
However, most are not objectively comparable with each other.
Possible reasons for this revolve around differences in the datasets and metrics used
during their evaluation or that the data input of the algorithms is fundamentally incomparable.
This lack of comparability renders the selection process of a suitable algorithm difficult. 

\section{Goals}

This thesis deals with a uniform comparison of selected plane detection algorithms.
Through this comparison, we aim to evaluate to what extent precise real-time plane detection is possible on affordable
hardware such as the \textit{Intel RealSense}. The answer to this question will also determine which algorithm is most suitable.
Note that this work focuses on plane detection in complete 3D environments.
Furthermore, we restrict ourselves to plane detection in indoor environments.
Thereby, the data is primarily recorded inside buildings.

\section{Structure}
The following chapter presents the basics or background information necessary for this work.

In Chapter~\ref{chap:Concept}, our concept of achieving the goals mentioned above is detailed. 
Therein, we prepare the evaluation by selecting suitable algorithms and datasets. A definition of \textit{real-time} closes the chapter.

Chapter~\ref{chap:impl} specifies the implementation details for the concept in the previous chapter. We outline the general system setup, the necessary steps included in the implementation of each algorithm, and the dataset modifications needed to conduct quantitative experiments.

The uniform comparison of the selected algorithms is conducted in Chapter~\ref{chap:eval}. Moreover, the results thereof
are presented and analyzed.

Based on the obtained results, we conclude in Chapter~\ref{chap:concl}. Lastly, the limitations thereof are
considered, and this work closes with the prospects of future research.
\end{document}
