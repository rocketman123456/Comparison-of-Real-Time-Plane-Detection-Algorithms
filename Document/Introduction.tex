\documentclass[main.tex]{subfiles}
\begin{document}
\chapter{Introduction}
\label{chap:Introduction}

% \textbf{\textcolor{red}{ist noch ein wenig scrambled. grundlegende toc:}}
% \begin{enumerate}
%     \item \textbf{Motivation}
%     \begin{itemize}
%         \item ebenen sind überall
%         \item plane detection ist voll oft genutzt und voll cool!
%         \item PD wird in diversen applikationen benutzt
%         \item manche dieser applikationen sind zeitlich eingeschränkt -> realtime
%         \item problem 1: sensor noise, grade indoors
%         \item ansatz 1: sensor improvement
%         \item problem 2: is' teuer, kann sich nicht jeder leisten
%         \item ansatz 2: bezahlbare sensoren -> intel RS, bester algorithmus / wie gut geht das ganze
%         \item problem 3: PDAs sind nicht vergleichbar
%     \end{itemize}
%     \item \textbf{Goals}
%     \begin{itemize}
%         \item plan: einheitlicher vergleich
%         \item ergbnis: wie gut geht PD auf OTS hardware, welcher algo ist der beste
%         \item einschänkungen auf indoor, ganze environments
%     \end{itemize}
%     \item \textbf{structure}
% \end{enumerate}
Man-made environments usually consist, to a large extent, of planar structures. 
Planar structures are often found in the basic structure of buildings, and much of the interior design has flat surfaces; 
even the reading of this work is likely to take place over a planar medium. 
Due to this large amount of planes in everyday environments, especially indoor environments, automatic detection of these planes is growing in relevance: 
Plane detection is an essential component in applications of numerous fields, including robotics~\cite{Zhang_Wang_Xianyu_Ziwei_Wei_2019}
and virtual or augmented reality (AR/VR)~\cite{Jurado_Jurado_Ortega_Feito_2021}. \textbf{\textcolor{red}{more sources!}}

Many of these systems operate under specific time constraints, which also influences the maximum available calculation time of plane detection. 
It is often referred to as real-time when such an application has to perform its calculations within a strict time limit. 
One of the main challenges of plane detection is that the sensors often have a high level of noise, which complicates the calculations, 
reduces the precision of the algorithms, or even renders the detection impossible altogether. 
The level of noise depends strongly on the movement of the sensors. 
Most AR/VR applications, as well as many SLAM algorithms (see Section~\ref{sec:bg-slam}), are performed in indoor environments. 
Due to fast movement in spatially confined environments, the precision of the recorded data decreases rapidly.
Moreover, large amounts of clutter are likely to occur in indoor environments, posing further challenges 
to the recording sensor as the observable surface gains complexity.

\paragraph{Problemabsatz so far}
A viable approach is, of course, to improve the sensors used to record the environment. However, as sensor precision increases, so does the price of the sensor. 
For instance, the AR device \textit{Microsoft HoloLens 2} internally performs plane detection as part of its spatial mapping process. 
The calculation time and precision are very good, which is to be expected for a price of around 3500. Another example would be sensors such as the \textit{Leica BLK360} 
laser scanner. Although it does not perform plane detection, it does provide very dense and precise point clouds. Even if this technology delivers precise data, 
the average user will not be able to afford it.

This lack of affordability raises the question as to what are the extents of real-time plane detection under the use of affordable hardware.
At a total of around 600 euros, the T256 and D455 from the \textit{Intel RealSense} series are an example of affordable sensor technology (see Section~\ref{sec:bg-intel}).
\textbf{\textcolor{red}{ist ein wenig kurz mmn}}


In addition to the used sensors, selecting an appropriate plane detection algorithm is important as well. 
Many algorithms have been designed over decades of research. 
However, most algorithms are not objectively comparable with each other. 
Possible reasons for this revolve around differences in the datasets and metrics used 
during their evaluation or that the data input of the algorithms is fundamentally incomparable.

\section{Goals}

This thesis deals with a uniform comparison of selected plane detection algorithms. 
Through this comparison, we aim to evaluate to what extent precise real-time plane detection is possible on affordable 
hardware such as the \textit{Intel RealSense}. The answer to this question will also determine which algorithm is most suitable. 
Note that this work focuses on plane detection in complete 3D environments. 
Furthermore, we restrict ourselves to plane detection in indoor environments. 
Thereby, the data is primarily recorded inside buildings. 

\section{Structure}
Chapter 2 präsentiert nötiges Hintergrundwissen für diese Arbeit.

Im Konzept (chapter 3) wird der grundsätzliche Leitfaden dieser arbeit Festgelegt.

Chapter 4 detailliert die implementations details.

Der einheitliche Vergleich wird in Kapitel 5 stattfinden. Darüber hinaus werden die Ergebnisse ausgewertet.

Wir werden das Fazit ziehen und somit die Frage dieser arbeit im letzten kapitel (chapter 6) beantworten.
\end{document}
