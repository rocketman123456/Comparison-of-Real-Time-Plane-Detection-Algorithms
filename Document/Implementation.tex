\documentclass[main.tex]{subfiles}
\begin{document}

\chapter{Implementation}
To ensure a uniform comparison of the algorithms selected in the concept, the external circumstances
must be identical for each algorithm. This chapter will detail the implementation thereof.

\section{System Setup}
It is necessary to perform all experiments on the same machine to ensure a consistent comparison.
We implement all algorithms and further architecture on a Lenovo IdeaPad 5 Pro,
which runs Linux Ubuntu 20.04.5. The laptop has an AMD Ryzen 7 5800H CPU and 16 GB of RAM.

We install the most recent ROS distribution, \textit{ROS Noetic Ninjemys}, as well as \textit{realsense-ros} with all additional dependencies.

% FIXME i think those footnotes with links were a bad idea

\section{Plane Detection Algorithms}
We implement RSPD and OPS using their respective open source implementations on GitHub\footnote{\href{https://github.com/abnerrjo/PlaneDetection}{{https://github.com/abnerrjo/PlaneDetection}}}
\footnote{\href{https://github.com/victor-amblard/OrientedPointSampling}{https://github.com/victor-amblard/OrientedPointSampling}}. Note that, while the implementation of RSPD is provided by the author,
we could not determine whether the user who uploaded his implementation of OPS is affiliated with \citeauthor{Sun_Mordohai_2019}.
Both methods are implemented in C++ and depend on the C++ linear algebra library \textit{Eigen}\footnote{\href{https://eigen.tuxfamily.org/index.php}{https://eigen.tuxfamily.org/index.php}}
and the C++ API of the  Point-Cloud Library\footnote{\href{https://pointclouds.org/}{https://pointclouds.org/}}, \textit{libpcl-dev}.

The authors of 3D-KHT, provide an implementation, in form of a Visual Studio project, on their website \footnote{\href{https://www.inf.ufrgs.br/~oliveira/pubs_files/HT3D/HT3D_page.html}
    {https://www.inf.ufrgs.br/~oliveira/pubs\_files/HT3D/HT3D\_page.html}}. Since the laptop we use does not run Windows, we use \textit{cmake-converter}
\footnote{\href{https://cmakeconverter.readthedocs.io/en/latest/use.html}{https://cmakeconverter.readthedocs.io/}} to convert
the solution to a cmake project we can build using \textit{make}.

\subsection*{OBRG}
To our knowledge, no open-source implementation is available for the algorithm.
We therefore use our own implementation.
% TODO after actual implementation of OBRG
We implement the method in python, heavily relying on someLib for computation of something.

\end{document}





