\documentclass[main.tex]{subfiles}
\begin{document}

\chapter{Implementation}
To ensure a uniform comparison of the algorithms selected in the concept, the external circumstances 
must be identical for each algorithm. This chapter will detail the implementation thereof.

\section{Hardware}
It is necessary to perform all experiments on the same machine to ensure a consistent comparison.
We implement all algorithms and further architecture on a Lenovo IdeaPad 5 Pro \footnote{\href{https://www.lenovo.com/de/de/laptops/ideapad/500-series/IdeaPad-5-Pro-16ACH6/p/88IPS501619}{Lenovo IdeaPad 5 Pro}},
which runs Linux Ubuntu 20.04.5.\\
The laptop has the following specifications.


\begin{mylist}
    \begin{itemize}
        \item CPU : AMD Ryzen 7 5800H @ 4.4GHz
        \item GPU : AMD RX Vega 8
        \item RAM : 16G
    \end{itemize}
    \end{mylist}

\section{Intel RealSense}
As stated in \ref{chap:Concept}, we use the open source implementation of Intel RealSense. 

\section{Plane Detection Algorithms}
\subsection*{RSPD}
An open source implementation, written in C++, is provided by the authors \citeauthor{Araújo_Oliveira_2020}\cite{Araújo_Oliveira_2020} and
can be found on GitHub \footnote{\href{https://github.com/abnerrjo/PlaneDetection}{\citeurl{Araújo_Oliveira_2020}}}.
We closely follow the instructions regarding the build process as described in the provided README.

\subsection*{OPS}
The authors of \textit{Oriented Point Sampling}, namely \citeauthor{Sun_Mordohai_2019}, do not provide their original implementation. 
However, a C++ implementation can also be found, again on GitHub \footnote{\href{https://github.com/victor-amblard/OrientedPointSampling}{\citeurl{Sun_Mordohai_2019}}}

\subsection*{3D-KHT}
\citeauthor{Limberger_Oliveira_2015}, the authors of 3D-KHT, provide an implementation, in form of a Visual Studio project, of their algorithm on their website \footnote{\href{https://www.inf.ufrgs.br/~oliveira/pubs_files/HT3D/HT3D_page.html}
{\citeurl{Limberger_Oliveira_2015}}}. Since the laptop we use does not run Windows, and therefore cannot build the project, we use a package called 
\textit{cmake-converter}\footnote{\href{https://cmakeconverter.readthedocs.io/en/latest/use.html}{https://cmakeconverter.readthedocs.io/}} to convert the solution to a cmake project we can build using \textit{make}.

\subsection*{OBRG}
To our knowledge, no open-source implementation is available for the algorithm, which was introduced by \citeauthor{Vo_Truong-Hong_Laefer_Bertolotto_2015}\cite{Vo_Truong-Hong_Laefer_Bertolotto_2015}.
We therefore use our own implementation.

%TODO
\textcolor{red}{sicherlich haben wir das geschafft. Weil C++ fucky wucky ist, haben wir das in python implementiert. Dazu haben wir diese hübschen libs genutzt...} 
\end{document}

 
 


