\documentclass[main.tex]{subfiles}
\begin{document}
\chapter*{Abstract}
% \begin{center}
%     \large{\textbf{Abstract}}
% \end{center}
\thispagestyle{empty}
Planar structures account for a significant portion of indoor man-made environments.
With advances in the field of Augmented Reality (AR), the automatic detection of planar surfaces has become essential for recent AR applications. Often, these applications operate under a strict temporal constriction, also referred to as \textit{real-time}. Naturally, this time restriction applies to the integrated plane detection algorithm as well. The technology that provides \textit{real-time} plane detection already exists. However, for different reasons, these devices are often not suitable for the average consumer. This motivates the utilization of consumer off-the-shelf hardware. Additionally, a suitable plane detection algorithm is needed. Decades of research yield a wide variety of different approaches. As these methods are predominantly evaluated scientifically, the real-world applicability poses an open question. Moreover, the inherent incomparability of most plane detection algorithms renders the selection of the most suitable non-trivial.\\
\phantom{test} This work evaluates the real-world applicability of \textit{real-time} plane detection algorithms. After superficially comparing current state-of-the-art plane detection algorithms and datasets, we select four algorithms, namely RSPD~\cite{Araújo_Oliveira_2020}, OPS~\cite{Sun_Mordohai_2019}, 3D-KHT~\cite{LimbergerOliveira2015HT3D}, and OBRG~\cite{Vo_Truong-Hong_Laefer_Bertolotto_2015}. Due to the lack of appropriate datasets, we select the \textit{2D-3D-S} dataset and compose the novel \textit{FIN} dataset. After giving a definition of \textit{real-time}, we perform experiments on both datasets, subsequently comparing the respective results. We conclude that 3D-KHT is the only \textit{real-time} applicable plane detection algorithm.

\end{document}
