\documentclass[main.tex]{subfiles}
\begin{document}


\chapter{Concept} \label{chap:Concept}

% \section*{Introduction}

This chapter covers our procedure of analysis.
We introduce the definition of real-time and the use case with respect to this work.
Plane detection algorithms are selected for comparison, as well as the metrics they will be judged upon.

\section{Scenario / UseCase}
Taking motivation from chapter \ref{chap:Introduction}, we focus on indoor environments during this work.
This includes the buildings we encounter in our normal lives, wether its the home we live in, the office we work in or a stripped-down
version of a building during construction.


\subsection{Used Sensors}
To be able to perform plane detection, we need special hardware that is able to accurately record the surroundings.
Numerous different cameras suffice for this task, of course varying in different aspects.
In this work, we use Intel RealSense technology, namely the T256 Tracking Camera and the D455 RGB-D Camera.
\textcolor{red}{probably more detail on the camera right? oder eher in den background}


In addition to the cameras, there is software that provdes a wide variety of usage. We are especially interested in detection of planes
in complete environments. For that reason, we use the ROS \footnote{\href{https://www.ros.org/}{ROS-Robot Operating System}} wrapper of
Intels Realsense software, \textit{realsense-ros} \footnote{\href{https://github.com/IntelRealSense/realsense-ros}{realsense-ros}}

Realsense-ros internally uses a SLAM(Simultaneous Mapping and Localization) algorithm called RTAB-MAP \cite{Labbé_Michaud_2019} for map-building.
RTAB-MAP is responsible for building a coherent map from a continuous stream of data that is being recorded and published by the two cameras.
It is worth noting, that the success of this work does not depend on the specific SLAM algorithm being chosen. We select RTAB-MAP because
it is already included in the realsense package and its reported performance suffices for this work, especially since we dont focus on SLAM
algorithms in this work.
\textcolor{red} {Furthermore noteworthy is the fact, that different algorithms eventually return different (kinds of) maps, which likely lead to different results in
    comparison to ours.}



\subsection{Real-Time}
To precisely define real-time, we need to consider the hardware restrictions of the Intel RealSense cameras
as well as the fps restrictions of RTAB-MAP.
RTAB-MAP publishes a new update of the current map with a default frame rate of 1Hz.
Thus, the plane detection algorithm needs to process a pointcloud in a similar frequency, taking up to an entire second.

We can reduce complexity further by taking the depth accuracy of the D455 into account. The RMS error of the
D455 is reported to be 2\% at 4 meters distance to the sensor, with a linear progression.
To maintain a dense and precise representation of our environment, we limit the detection of planes to a
radius of 4 meters from the current position.

\section{Selection Plane Detection Algorithms}
First we need to assert the comparability between the algorithms introduced in \ref{chap:Background}.
We report necesary criteria to both shorten the list of algorithms, as well as verify comparability.

\subsection*{Type of Input}
Popular representations, which the recorded environment can take the form of, can be grouped into three main categories of input:
\begin{itemize}
    \item \textit{unorganized} or \textit{unstructured point cloud}
    \item \textit{organized} or \textit{structured point cloud}
    \item \textit{(depth-) image}
\end{itemize}
As stated before, we focus on the detection of planar structures in the entirety of a scene, rather than just singular segments thereof.
In addition, only the first type of input offers a persistent view on the recorded environment.
For that reason, we disregard all algorithms which do not expect an unorganized point cloud as input.




\subsection*{Determinism}
ND methoden, zb DL basierende haben grundsätzlich ein gewisses level an bias, welcher stark von der Wahl der Trainingsdaten abhängt.
Ein weiterer grund gegen die Benutzung von Learning-basierenden methoden ist, dass wir nicht von einer (rechenstarken) GPU ausgehen können.

\textcolor{red}{ransac ? OPS nutzt halt ransac}

\subsection{Paremeters, additional necessitites}
Ebenfalls wichtig ist, ob ein algorithmus zusätzlich zum input noch weiteres wissen benötigt. 3DKHT ist stark von dem Level der
octree subdivision abhängig, welches (ohne weitere Änderungen) predefined ist.

\subsection*{Availability}
Zuletzt werden wir die verfügbarkeit als schwaches kriterium festlegen.
Grundsätzlich muss ein algorithmus auf dem selben System, wie die anderen auch implementiert sein, damit die ausführende/underlying hardware
als Faktor ausgeklammert werden kann. Aus dem Grund können algorithmen, welche wenig bis gar nicht beschrieben werden, nicht in diesen Vergleich
aufgenommen werden.

\subsection{RSPD - Robust Statistics Approach for Plane Detection}
\subsection{OPS - Oriented Point Sampling}
\subsection{3DKHT - 3-D Kernel-based Hough Transform}
der bums braucht halt fixe parameter 
\textcolor{red}{vielleicht kann ich den dynamisch-ish machen in abhängigkeit von der dimension der input wolke}
\subsection{OBRG - Octree-based Region Growing}
\subsection{PEAC - Probabilistic Agglomerative Hierarchical Clustering}
hier ist ne tabelle:
\begin{table}[!h]
    \centering
    \begin{tabular}{|l|l|l|l|l|}
        \hline
                       & \textbf{Input Data} & \textbf{Determinism} & \textbf{Parameter Dependency} & \textbf{Availability} \\ \hline
        \textbf{RSPD}  & unordered PC        & Deterministic        & independent                   & open source           \\ \hline
        \textbf{OPS}   & unordered PC        & mostly Deterministic & independent                   & open source           \\ \hline
        \textbf{3DKHT} & unordered PC        & mostly Deterministic & very dependent                & open source           \\ \hline
        \textbf{OBRG}  & unordered PC        & Deterministic?       & ?                             & not available         \\ \hline
        \textbf{PEAC}  & ordered PC          & Deterministic        & ?                             & avalable              \\ \hline
    \end{tabular}
    \caption{Selected Plane Detection Algorithms}
    \label{tab:my-table}
\end{table}

\subsection*{Fazit}
wir vergleichen RSPD, OPS und 3D-KHT. \textcolor{red}{depending on my success in implementing OBRG myself we will add OBRG to the comparison.}
We exclude PEAC beacuse it excpects an ordered point cloud as input.  

\end{document}