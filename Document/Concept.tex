\documentclass[main.tex]{subfiles}
\begin{document}


\chapter{Concept} \label{chap:Concept}


\begin{figure}[H]
    \centering
    \includegraphics[width=15 cm]{images/concept_specific.png}
    \caption[Concrete Concept Graphic]{The procedure of the plane detection process. The specialized sensor records data ([1]), which is passed to
        a SLAM algorithm ([2]). After map assembly, a point cloud is handed to a plane detection algorithm ([3]).
        The detected planes are given to a use-case-specific application ([4]).}
    \label{fig:concept}
\end{figure}

% FIXME direkt die spezifische nehmen und im folgenden erklären wie ich dazu komme
% FIXME ebenen zurück in die scene / transformation / tracking oder so 
% FIXME formale korrektheit der grafik: 
% \section{Scenario / UseCase}

% Within a given use case or scenario that involves the detection of planes, the user records the surroundings with a camera. The raw data is then either
% directly, or after some preprocessing steps, passed to a plane detection algorithm, which then hands the detected planes to an application for further use.
% This application could use those planes to define the playable area in an augmented reality (AR) video game or alternatively build a digital floor
% plan, or 3D model, of an apartment.

% Especially in scenarios where the user moves through the environment, the time between recording and the planes reaching the application is crucial.
% If an autonomous robot moves through an apartment complex, the delay must be as short as possible to avoid a collision with a wall the robot has not yet detected.
% The problem can therefore be defined in such a way that planes must be detected in real-time and at the same time be sufficiently precise for the given use case.

% % To optimize the entire process, shown in Figure~\ref{fig:concept}, this work will aim to answer the question of to what extent precise real-time plane detection is feasible. 
% % NOTE alternative: 
% To optimize the entire process, shown in \ref{fig:concept}, this work is focused on the plane detection step (green).

% Furthermore, we focus on indoor environments, as motivated by Chapter~\ref{chap:Introduction}.
% This includes the buildings we encounter in our normal lives, whether it is the home we live in, the office we work in, or a stripped-down version of a building during construction.

\paragraph*{introduction}
\begin{enumerate}
    \item aktueller stand: gute und schnelle ebenenfindung wird oft gebraucht, gibt es auch schon/ist möglich
    \item problem: oft sind die speziellen sensoren sehr kostenspielig
    \item Daher die frage: (wie gut) ist das ganze auf bezahlbarer hardware möglich?
    \item Nötig, um die Frage zu beantworten: 
        \begin{itemize}
            \item Welche Kamera(s)?
            \item Welcher algorithmus?
            \item Was heisst "real-time" überhaupt?
        \end{itemize} 
    \item Problem an letzterem: nicht möglich einen einheitlich besten algorithmus auszuwählen, da \dots{}
    \item Lösung: wir wählen algorithmen aus und vergleichen diese einheitlich um die frage aus 3. zu beantworten
\end{enumerate}

% Viele Anwendungsfälle bauen auf das finden von ebenen in realen strukturen auf. Es werden spezielle sensoren entwickelt, welche extrem präzise daten der aufgenommenen umgebung liefern können.
% Mit dieser präzision steigt natürlich auch der preis, was sich natürlich nicht jeder leisten kann. Ein Nutzer möchte und kann sich wahrscheinlich nicht einen 4000 Sensor leisten, nur um AR spiele in seinem Wohnzimmer zu spielen.
% Dazu kommt oft eine zeitliche komponente. Wenn sich der nutzer durch die umgebung bewegt möchte dieser nicht 






\section{Intel RealSense}
\begin{itemize}
    \item Wir müssen zuerst sensoren auswählen, mit denen wir die umgebung aufnehmen
    \item Da, wie vorher angesprochen, der preis des sensors oft ein problem ist, wählen wir eine relativ billige (im vergleich)
    \item die intel sensoren sind vergleichsweise bezahlbar. 
    \item genauer gesagt nutzen wir \dots{}
    \item Zu den sensoren wird eine kostenfreie software bereit gestellt
    \item Über diese software lassen sich die kameras ansteuern. dazu ist in dieser software ein slam algorithmus namens rtabmap implementiert
    \item mit rtabmap können wir den strom aus rohdaten zu einer bestehenden karte verarbeiten, was uns ermöglicht ebenen der kompletten umgebung zu finden anstatt nur von dem aktuellen blickwinkel
\end{itemize}

% We need specialized sensors, namely cameras, to record the environment and obtain data to detect planes therein (see [1] in Figure~\ref{fig:concept}).


% In this work, we use the Intel RealSense T265 Tracking Camera and the Intel RealSense D455 RGB-D Camera, because of their compatibility.
% The T265 can be used in combination with any depth camera from the D400 series, thus combining the advantages of both; The depth perception of the D455 and the wide FOV of the T265.
% Furthermore, due to built-in IMUs, both cameras can perform all position calculations which reduces the load on the hosting device, e.g., laptop or microcontrollers like Raspberry PI's.

% Since we are especially interested in the detection of planes in complete environments, it is necessary to be able to build a map out of the data the cameras
% record continuously. For that purpose, we use a SLAM algorithm ([2] in Figure~\ref{fig:concept}).
% In addition to the cameras, the Intel RealSense software provides a way to perform map assembly, see step 2 of Figure~\ref{fig:concept}.%\textcolor{red}{siehe concept figure, step 2}.
% We integrate \textit{realsense-ros}, the ROS wrapper of Intel RealSense, into our process of plane detection.

% Realsense-ros internally uses a SLAM algorithm called RTAB-MAP \cite{Labbé_Michaud_2019} for map-building.
% RTAB-MAP is responsible for building a coherent map from a continuous stream of data that is being recorded and published by the two cameras.
% It is worth noting, that the success of this work does not depend on the specific SLAM algorithm being chosen. We select RTAB-MAP because
% it is already included in the RealSense package and its reported performance suffices for this work, primarily since we don't focus on SLAM
% algorithms in this work.

\section{Selection of Plane Detection Algorithms}\label{sec:pdaselection}
This section deals with the selection of appropriate plane detection algorithms. 
First, we define meaningful criteria against which the algorithms are compared to allow objective selection.

\subsection{Criteria}
\textcolor{red}{disclaimer: die gründe für oder gegen bestimmte sachen in dieser section werden noch nach hinten verschoben und durch eine erklärung ersetzt ist grade nur nicht die haupt baustelle}
% \subsection{Type of Input}
\paragraph{Type of Input}\label{par:input}
Usually, the data representation of the recorded environment passed to the plane detection algorithm falls under one of three categories:
\begin{itemize}
    \item \textit{unorganized} or \textit{unstructured point cloud} (UPC)
    \item \textit{organized} or \textit{structured point cloud} (OPC)
    \item \textit{(depth-) image} (D-/I)
\end{itemize}

As stated before, we focus on detecting planar structures in the entire environment rather than just distinct segments thereof.
In addition, only the unorganized/unstructured point clouds offer a complete view of the recorded environment.
% NOTE kommt später: Therefore, we disregard all algorithms which do not expect an unorganized point cloud as input.


% \subsection{Detected Plane Format} \label{subsec:planeformat}
\paragraph{Detected Plane Format} \label{subsec:planeformat}
Which specific representation the detected planes take the form of is also essential.
If no uniform output type can be determined, consequently, no uniform metric for comparison can also be found.
Since the algorithms process point clouds, we choose to stay within the realm of points, i.e., an arbitrary plane should be
represented by the set of points included in the plane (inliers).
The representation, being a list of points, enables further processing of the detected planes.
A list of points would, in contrast to some plane equation, enable us to detect holes in planes, e.g.,
an open door or window, which can be helpful in any use case involving remodeling architectural elements.
It also allows further filtering of planes based on a density value that we can calculate over the bounding
box and the number of points, e.g., removing planes with a density lower than a certain threshold.


% \subsection{Learning based}\label{subsec_learning_based}
\paragraph{Learning based}\label{subsec_learning_based}
% FIXME erklären was learning based methods machen, unterschieden zu non-learning based!!
Learning-based methods, e.g. Deep learning, generally have varying levels of bias, depending on the training data.
Another reason against the use of learning-based methods is that we choose not to require a GPU to replicate our findings.

%  FIXME Gründe dagegen in selection schieben !!

% \subsection{Availability}
\paragraph{Availability}
% FIXME Gründe und erklärung! zb, müssen alle auf dem selben system laufen, keine patente oä, das wissen dahinter ist frei zugänglich 
%  sodass man die im zweifel zumindest selber implementieren könnte
Lastly, we include the availability of an algorithm in our set of criteria.
Each algorithm to be compared needs to run on the same system to exclude the underlying hardware as a factor from any experiments.\\

\subsection{Plane Detection Algorithms}

% Diese section befässt sich mit der auswahl der plane detection algorithmen. Eine umfassende Literaturrecherche des Gebiets der plane detection 
% ergibt eine Liste (vgl. Table~\ref{tab:algos}) von algorithmen. Diese werden hier zunächst in dedizierten paragraphen umrissen. 
% Im folgenden werden diese algorithmen verglichen und anschliessend eine auswahl, basierend auf den zuvor ausgewählten kriterien, für den weiteren Verlauf dieser Arbeit getroffen.

A list of state-of-the-art algorithms is compiled through comprehensive research of the current literature on plane detection (see Table~\ref{tab:algos}).
Each algorithm is outlined in the following dedicated paragraphs. In the following, these algorithms are compared and then selected based on the previously selected criteria for the further course of this work.

\paragraph{RSPD - Robust Statistics Approach for Plane Detection}
RSPD expects an unorganized point cloud as input. The detected planes are written to file as either a list of inliers or
represented by a center point, a normal vector and UV extents of the plane.
As explained above \ref{subsec:planeformat}, we select the inliers as the primary plane representation.

\paragraph{OPS - Oriented Point Sampling}
OPS also takes an unorganized point cloud as input. Like RSPD, the default representation of a detected plane is
a tuple of normals vector and center coordinate, while also reporting the orientation.
without affecting the algorithm, we modify the output format to instead include the inliers.

\paragraph{3DKHT - 3-D Kernel-based Hough Transform}

\textcolor{red}{das hier wird sich noch ändern - habe das hier vor dem deepdive zwecks BG geschrieben}

With 3D-KHT, as with other octree-based methods, the performance depends, to some degree, on the level of subdivision.
In the provided implementation, the octree keeps dividing until either the number of points in the current node is lower than a set minimum, or the
level of an octree node is higher than a predefined maximum.

A total of six different presets of parameters are included with the official implementation.
Since this work does not focus on the evaluation and analysis of a single method, we performed experiments with all presets on Area 3 of the S3DIS\cite{armeni_cvpr16} dataset, the
results thereof can be seen in Figure ~\ref{fig:3dkht_params}
The two leftmost values for the maximum octree subdivision seem to yield the best results compared to the other values.
Because we expect a greater subdivision level to yield better results on scenes of larger dimensions, we choose to perform
all further experiments with an octree subdivision value of two.


\begin{figure}[H]
    \centering
    \includegraphics[width=15 cm]{images/params_3dkht.png}
    \caption[3D-KHT Parameter Benchmark Results]{Results of 3D-KHT for different pre-set subdivision values}
    \label{fig:3dkht_params}
\end{figure}


\paragraph{OBRG - Octree-based Region Growing}
\paragraph{PEAC - Probabilistic Agglomerative Hierarchical Clustering}
\paragraph{CAPE - Fast Cylinder and Plane Extraction}
\paragraph{SCH-RG - Plane Extraction using Spherical Convex Hulls}
\paragraph{D-KHT - Hough Transform for Real-Time Plane Detection}
\paragraph{DDFF - Depth Dependent Flood Fill}
\paragraph{PlaneNet}
\paragraph{PlaneRecNet}
\paragraph{PlaneRCNN}


\subsection{Conclusion: Plane Detection Algorithms}
\begin{table}[H]
    \centering
    \begin{tabular}{|c|c|c|c|c|c}
        \hline
                                                                         & \textbf{Input Data} & \textbf{Plane Format} & \textbf{Learning-Based} & \textbf{Availability} \\ \hline
        \textbf{RSPD} \cite{Araújo_Oliveira_2020}                        & UPC                 & inliers               & N                       & Y                     \\ \hline
        \textbf{OPS} \cite{Sun_Mordohai_2019}                            & UPC                 & inliers               & N                       & Y                     \\ \hline
        \textbf{3DKHT} \cite{Limberger_Oliveira_2015}                    & UPC                 & inliers               & N                       & Y                     \\ \hline
        \textbf{OBRG} \cite{Vo_Truong-Hong_Laefer_Bertolotto_2015}       & UPC                 & inliers               & N                       & N                     \\ \hline
        \textbf{PEAC} \cite{Feng_Taguchi_Kamat_2014}                     & OPC                 & inliers               & N                       & Y                     \\ \hline
        \textbf{CAPE} \cite{Proença_Gao_2018}                            & OPC                 & normal, d             & N                       & Y                     \\ \hline
        \textbf{SCH-RG} \cite{Mols_Li_Hanebeck_2020}                     & OPC                 & inliers?              & N                       & N                     \\ \hline
        \textbf{D-KHT}  \cite{Vera_Lucio_Fernandes_Velho_2018}           & DI                  & inliers               & N                       & Y                     \\ \hline
        \textbf{DDFF} \cite{Roychoudhury_Missura_Bennewitz_2021}         & DI                  & inliers               & N                       & Y                     \\ \hline
        \textbf{PlaneNet} \cite{Liu_Yang_Ceylan_Yumer_Furukawa_2018}     & I                   & normal, d             & Y                       & Y                     \\ \hline
        \textbf{PlaneRecNet} \cite{Xie_Shu_Rambach_Pagani_Stricker_2022} & I                   & ? / -                 & Y                       & Y                     \\ \hline
        \textbf{PlaneRCNN} \cite{Liu_Kim_Gu_Furukawa_Kautz_2019}         & I                   & normal + ?            & N                       & Y                     \\ \hline
    \end{tabular}
    \caption{Plane Detection Algorithms}
    \label{tab:algos}
\end{table}

To effectively compare the presented algorithms, the data on which each algorithm performs the plane detection should ideally be the same.
Considering algorithms that run on anything other than UPC, would necessitate finding a data set that includes equivalent point clouds for both the structured and the unstructured case.
Since these algorithms would disregard the global structure of the point cloud, we deem them not feasible for our use case and thus exclude them from our evaluation.

% FIXME seems repeating to me, having a paragraph per criteria to reference them here in one single sentence
We exclude learning-based methods for the reasons previously stated in Subsection~\ref{subsec_learning_based}.

For an even comparison, the detected planes would have to be in the same format because, even for the same plane, representations could very well lead to different results, e.g., a plane in cartesian form compared to the same plane, described by its inliers.
Asserting comparability, we exclude all methods which do not offer a plane representation by inliers.

Furthermore, writing our own implementation of methods for which no implementation is available or for which the respective publication does not focus on the implementation details would go beyond the scope of this work.

Finally, we end up with, and thus include, the following plane detection algorithms in our evaluation:

\begin{itemize}
    \item RSPD
    \item OPS
    \item 3D-KHT
    \item OBRG
\end{itemize}


\section{Definition Real-Time}\label{sec:realtime}
To determine whether or not an algorithm runs in real-time, we must first define the meaning of real-time.

We have to consider possible hardware limitations, data flow, and simply
how often it is needed to perform calculations in correspondence with the given use case.

%TODO Gar nicht mehr unbedingt nötig, wenn die obere grenze eh der SLAM ist, oder? 
%  The D455 has a depth frame rate of up to 90, while the T265 only achieves a maximum frame rate of 30
The recorded raw data is not directly sent to the plane detection algorithm but instead given to RTAB-MAP, which then performs
calculations to update and publish the map.
Therefore, the upper limit is the frequency of how often RTAB-MAP publishes those updates, which by default is once per second.
According to this upper limit, we consider an algorithm \textit{real-time applicable}, if it achieves an average frame
rate of minimum 1, e.g., the algorithm manages to process the entire point cloud and detect all planes within one second.

%TODO  Eig nur needed wenn die algos langsamer sind als 1s ODER sollte die cloud extrem wachsen kann man sich auf die 6 meter beschränken, erstmal aber nicht}\\
% \subsection*{reduktion (opt)}
% We can reduce complexity further by taking the specifications (background)
% of the D455 into account. The RMS error of the D455 is reported to be 2\% at 4 meters distance to the sensor.
% Furthermore, the ideal distance is stated to range between $0.6 - 6$ meters.
% To maintain a dense and precise representation of our environment, we therefore limit the detection of planes to a
% radius of 6 meters from the current position.

\section{Summary}
Many applications have constraints in the form of a temporal component. Applications that include plane detection
are no exception. In addition to time constraints, good quality is usually tightly coupled to expensive sensors.
To evaluate to what extent it is possible to perform precise plane detection with a real-time constraint on off-the shelf hardware,
we compare selected algorithms.

\end{document}