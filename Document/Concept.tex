\documentclass[main.tex]{subfiles}
\begin{document}


\chapter{Concept} \label{chap:Concept}


\begin{figure}[H]
    \centering
    \includegraphics[width=15 cm]{images/concept_specific.png}
    \caption[Concrete Concept Graphic]{The procedure of the plane detection process. The specialized sensor records data ([1]), which is passed to
        a SLAM algorithm ([2]). After map assembly, a point cloud is handed to a plane detection algorithm ([3]).
        The detected planes are given to a use-case-specific application ([4]).}
    \label{fig:concept}
\end{figure}

Many AR and VR Systems integrate plane detection into their software, some use it only to calculate the ground floor while others use plane detection to
build a smaller model of the environment. Die genauen Anwengunsmöglicheiten sind hierbei jedoch endlos.
Figure~\ref{fig:concept} shows a generic block diagram of such a VR/AR system including plane detection.
% FIXME REALTIME NOW MISSING FROM INTRODUCTION
In general, the environment is continuously recorded by a specialized sensor which is usually a camera([1]). A SLAM algorithm then integrates the new data into its already existing map([2]). The map, in form of a point cloud,
is subsequently passed to a plane detection algorithm([3]). The algorithm performs the necessary steps to detect all planes inside the current map and passes the planes to the application([4]).
The application would then further process those planes, e.g., by creating a live visualization of them or by assisting the movement of visually impaired people~\cite{Carranza_Estrella_Zaidi_Carranza_2021}.

When creating such a system, the choice of PDA is naturally of great importance. The problem is that most published algorithms are not inherently comparable.
Often different datasets or metrics are used, which precludes comparison by quantification. Alternatively, algorithms are not comparable by internal functionality
because many methods require different inputs, and the format of the planes differs accordingly. All in all, selecting a single "best" algorithm is impossible solely based on the metrics presented in their
respective work.

To answer the question of which algorithm is best and whether it is real-time capable, we make a unified comparison of PDAs.
To perform this evaluation, we need the following things:

\begin{enumerate}
    \item \label{enum:pda}Appropriate plane detection algorithms,
    \item \label{enum:ds} a useful dataset \textit{and}
    \item \label{enum:rt} a definition of \textit{real-time}.
\end{enumerate}
The following sections are dedicated to them.


% Roter Faden:
% \begin{itemize}
%     \item generic view on AR + VR system
%     \item Describe the building blocks in there
%     \item Point out that the decision on the "best" PDA is not clear
%     \item Tell me why that is not clear!
%     \item Get me going on those PDA
%           \begin{itemize}
%               \item describe my view on the world
%               \item show marco that table
%               \item describe what these values mean and why they are important
%           \end{itemize}
% \end{itemize}

\section{Selection of Plane Detection Algorithms}\label{sec:pdaselection}
% SECTION get me going on those PDA 
% FIXME describe my view on the world
% FIXME show marco that table
% FIXME describe what these values mean and why they are important
%!SECTION - 

Since, as already noted, most algorithms differ in certain aspects, it is not possible to compare them all uniformly. Furthermore, not all algorithms have the same motivation and therefore focus on different things.
For example, testing an algorithm like \textit{Underwater SLAM}~\cite{6913908} for performance in small indoor environments would be pointless.
It is, therefore, necessary to first define objective criteria to superficially determine which algorithm seems to be relevant for the context of this work.


\subsection{Criteria}
In the following paragraphs, we define and outline appropriate criteria for the objective assessment of plane detection algorithms.

\paragraph{Type of Input}\label{par:input}
The first criterion is the type of input expected by a plane detection algorithm.
Usually, the data representation of the recorded environment falls into one of three categories:
\begin{itemize}
    \item \textit{unorganized} or \textit{unstructured point cloud} (UPC)
    \item \textit{organized} or \textit{structured point cloud} (OPC)
    \item \textit{(depth-) image} (D-/I)
\end{itemize}
% FIXME in den Background verschieben ,hier nur kurz den unterschied umreissen 

OPC and UPC both describe point clouds in the cartesian coordinate system. The primary difference is that the 3D coordinates inside
an organized point cloud are saved in a 2D Grid, while the unorganized cloud resembles an unsorted 1D array.
Like OPC, depth images are a 2D grid of values. However, in contrast to the 3D coordinates of an OPC, the data points of depth images
are the distances to the sensor.


% As stated before, we focus on detecting planar structures in the entire environment rather than just distinct segments.
% In addition, only the unorganized/unstructured point clouds offer a complete view of the recorded environment.


% \subsection{Detected Plane Format} \label{subsec:planeformat}
\paragraph{Detected Plane Format} \label{subsec:planeformat}
Which specific representation the detected planes take the form of is also essential.
If no uniform output type can be determined, consequently, no uniform metric for comparison can also be found.
Since the algorithms process point clouds, we choose to stay within the realm of points, i.e., an arbitrary plane should be
represented by the set of points included in the plane (inliers).
The representation, being a list of points, enables further processing of the detected planes.
A list of points would, in contrast to some plane equation, enable us to detect holes in planes, e.g.,
an open door or window, which can be helpful in any use case involving remodeling architectural elements.
It also allows further filtering of planes based on a density value that we can calculate over the bounding
box and the number of points, e.g., removing planes with a density lower than a certain threshold.


\paragraph{Learning based}\label{subsec_learning_based}
Some methods are learning-based, e.g., they use deep learning to detect planes in point clouds or images.
Their ability to generalize depends on the choice of training data. If an algorithm is perfectly trained on
a dataset that consists of only tables, the algorithm would find all table tops but might fail to detect planes that do not have a certain number
of legs attached to them.
\textcolor{red}{todo:elaborate}
% FIXME elaborate

% \subsection{Availability}
% \paragraph{Availability / Reproducability?}
% Lastly, we include the availability of an algorithm in our set of criteria.
% Since we need to conduct our experiments on the same machine, it is necessary that we obtain an implementation of a given method.
% We generally consider a method as \textit{available} if one of the following conditions is met:
% \begin{itemize}
%     \item A corresponding implementation is publicly available \textit{or}
%     \item the paper outlines the method in a way that enables self-implementation, i.e., by providing pseudocode, elaborate figures and/or thorough description.
% \end{itemize}

\subsection{Plane Detection Algorithms}

A list of state-of-the-art algorithms is compiled through comprehensive research of the current literature on plane detection (see Table~\ref{tab:algos}).
\textcolor{red}{explanation of table}

Im folgenden werden aus den zuvor aufgestellten kriterien die für diese arbeit sinnvollsten werte(?\textcolor{red}{"ich nehme UPC aus {UPC, OPC, DI}... idk wie ich das nennen soll}) ausgewählt und anhand dessen unpassende algorithmen von der evaluierung ausgeschlossen.

\begin{table}[H]
    \centering
    \begin{tabular}{c|c|c|c}
        \textbf{Plane Detection Algorithm}                               & \textbf{Input Data} & \textbf{Plane Format} & \textbf{Learning-Based} \\ \hline%& \textbf{Availability} \\ \hline
        \textbf{RSPD} \cite{Araújo_Oliveira_2020}                        & UPC                 & inliers               & N                       \\ %& Y                     \\ \hline
        \textbf{OPS} \cite{Sun_Mordohai_2019}                            & UPC                 & inliers               & N                       \\ %& Y                     \\ \hline
        \textbf{3DKHT} \cite{Limberger_Oliveira_2015}                    & UPC                 & inliers               & N                       \\ %& Y                     \\ \hline
        \textbf{OBRG} \cite{Vo_Truong-Hong_Laefer_Bertolotto_2015}       & UPC                 & inliers               & N                       \\ %& Y                   \\ \hline
        \textbf{PEAC} \cite{Feng_Taguchi_Kamat_2014}                     & OPC                 & inliers               & N                       \\ %& Y                     \\ \hline
        \textbf{CAPE} \cite{Proença_Gao_2018}                            & OPC                 & normal, d             & N                       \\ %& Y                     \\ \hline
        \textbf{SCH-RG} \cite{Mols_Li_Hanebeck_2020}                     & OPC                 & inliers               & N                       \\ %& N                     \\ \hline
        \textbf{D-KHT}  \cite{Vera_Lucio_Fernandes_Velho_2018}           & DI                  & inliers               & N                       \\ %& Y                     \\ \hline
        \textbf{DDFF} \cite{Roychoudhury_Missura_Bennewitz_2021}         & DI                  & inliers               & N                       \\ %& Y                     \\ \hline
        \textbf{PlaneNet} \cite{Liu_Yang_Ceylan_Yumer_Furukawa_2018}     & I                   & normal, d             & Y                       \\ %& Y                     \\ \hline
        \textbf{PlaneRecNet} \cite{Xie_Shu_Rambach_Pagani_Stricker_2022} & I                   & reconstructed scene   & Y                       \\ %& Y                     \\ \hline
        \textbf{PlaneRCNN} \cite{Liu_Kim_Gu_Furukawa_Kautz_2019}         & I                   & normal, d             & Y                       \\ %& Y                     \\ \hline
    \end{tabular}
    \caption{Plane Detection Algorithms}
    \label{tab:algos}
\end{table}

Addressing the criterion of input type, we are only interested in performing plane detection in complete environments.
Because unorganized point clouds are not limited in their dimension, they are more suitable for capturing entire environments.
We hereby consider organized point clouds or images inappropriate because they do not offer a complete view
on a scene.
We hereby exclude  \textit{PEAC, CAPE, SCH-RG, D-KHT, DDFF, PlaneNet, PlaneRecNet} and \textit{PlaneRCNN} from our evaluation.

% Considering algorithms that run on anything other than UPC, would necessitate finding a data set that includes equivalent point clouds for both the structured and the unstructured case. % FIXME or calculate
% Alternatively, it would require calculating equivalent UPCs for each OPC which is beyond the scope of this work, due to the dramatic increase in complexity.
% Since these algorithms would disregard the global structure of the point cloud, we deem them not feasible for our use case and thus exclude them from our evaluation.

Secondly, the detected planes need to be in the same format because, even for the same plane, different representations could very well lead to different results.
Assume a plane in cartesian form and a plane represented by its inliers. The calculated metrics may differ significantly because the plane in cartesian form is infinitely dense.
In contrast, the plane described by its inliers allows for holes and non-rectangular shapes, e.g., doorways or a round table.
We thereby determine \textit{inliers} as the preferred plane format and exclude all methods which do not comply, namely \textit{CAPE, PlaneNet, PlaneRecNet}, and \textit{PlaneRCNN}.


Finally, we end up with, and thus include, the following plane detection algorithms in our evaluation:

\begin{itemize}
    \item \textbf{RSPD}
    \item \textbf{OPS}
    \item \textbf{3D-KHT}
    \item \textbf{OBRG}
\end{itemize}

\section{Datasets}
% FIXME
% eigene arbeit nicht unterschlagen!
% datensätze kurz umreissen
% analog zu PDA section
% "aber keiner hat eine temporale komponente, daher müssen wir einen eigenen aufnehmen" 

Through extensive research of current literature, we compiled a list of popular datasets (see Table~\ref{tab:datasets}).

Wir haben bereits festgelegt, dass unorganized point clouds der benötigte input für unsere algorithmen ist. Dazu fokussieren wir uns in dieser arbeit auf ebenenfindung in echten umgebungen.
Anhand dieser Anforderungen fallen die meisten Datensätze raus und nur \textit{S3DIS} und \textit{Leica} bleiben übrig. Außerdem betrachten wir hier nur indoor environments, weswegen \textit{leica}
auch weg fällt und \textit{S3DIS} als einzigen übrig lässt.

Die ground truths von S3DIS sind zwar im format einer unorganized point cloud, wir können sie für diese arbeit dennoch nicht benutzen,
da diese keine ebenen darstellen, sondern segmentierte objekte.
Wir werden daher selber eine ground truth für den Datensatz erstellen (siehe~\ref{sec:gtseg}).

Auch wenn wir nun S3Dis für eine evaluierung nutzen können, hat dieser keine temporale komponente, e.g. baut dieser sich nicht über zeit auf.
unserer literaturrecherche entsprang kein datensatz, der den oben genannten kriterien entspricht und eine auf ebenen fokussierende ground truth angibt.
Consequently, nehmen wir einen eigenen datensatz in der Fakultät für Informatik(FIN) der Otto-von-Guericke Universtität Magdeburg auf.



% \subsection{FIN Experiment}
% We record an incrementally growing dataset in the Faculty of Computer Science at Otto-von-Guericke University Magdeburg.
To perform a thorough comparison between the FIN and S3DIS, and, subsequently, between the static and the dynamic dataset, we record a scene for each of the following scene types:
\begin{itemize}
    \item office
    \item conference room
    \item auditorium
    \item hallway
\end{itemize}

We focus on these four scene types because they are the most common in a real environment.
The recorded point clouds can be seen in Figure~\ref{fig:fin}.

Furthermore, since this is a novel dataset, we create a ground truth. The details thereof are explained in Section~\ref{sec:finimpl}.

\begin{figure}[H]
    \begin{subfigure}{0.5\textwidth}
        \centering
        \includegraphics[width=.9\linewidth]{images/307.png}
        \caption[Dynamic Dataset - auditorium]{}
        \label{fig:fin307}
    \end{subfigure}
    \begin{subfigure}{0.5\textwidth}
        \centering
        \includegraphics[width=.9\linewidth]{images/333.png}
        \caption[Dynamic Dataset - conference room]{}
        \label{fig:fin333}
    \end{subfigure}
    \begin{subfigure}{0.5\textwidth}
        \centering
        \includegraphics[width=0.9\linewidth]{images/425.png}
        \caption[Dynamic Dataset office]{}
        \label{fig:fin425}
    \end{subfigure}
    \begin{subfigure}{0.5\textwidth}
        \centering
        \includegraphics[width=0.9\linewidth]{images/hallway.png}
        \caption[Dynamic Dataset office]{}
        \label{fig:finhw}
    \end{subfigure}
    \caption[Dynamic Datasets]{The recordings for each scene type: (a)auditorium, (b) conference room, (c) office and (d) hallway.}
    \label{fig:fin}
\end{figure}

% Running \textit{realsense-ros} and holding our cameras, we walk through the aforementioned parts of the building while scanning to the best of our ability.
% We save each incremental map update to a file for later usage.

% Since no ground truth exists for a novel dataset like this, we create a set of ground truth planes $gt_{end}$ for only the most recent update of each scene, e.g., for the entire recording.
% To prepare for the evaluation of a map $m_t$ at a given time $t$, we crop all planes in $gt_{end}$ by removing all points that are not present in $m_t$, as shown in
% Figure~\ref{fig:dynGT}.
% We speed up this expensive process by employing a KD-Tree neighbor search with a small search radius since we only need to know whether a certain point is present or not.
% % Furthermore, we remove planes from the ground truth if the number of included points falls short of a threshold. 
% \begin{figure}[H]
%     \centering
%     \includegraphics[width=15 cm]{images/dynamic_eval.png}
%     \caption[Dynamic Ground Truth Generation]{Dynamic ground truth generation. All planes that are included in \textit{Ground Truth} are cropped depending on
%         the available point cloud at each time \textit{t} }
%     \label{fig:dynGT}
% \end{figure}


\section{Definition Real-Time}\label{sec:realtime}
% FIXME "wirkt etwas komisch hier. bin mir noch unsicher." 
To determine whether or not an algorithm runs in real-time, we must first define the meaning of real-time.

We have to consider possible hardware limitations, data flow, and simply
how often it is needed to perform calculations in correspondence with the given use case.

%TODO Gar nicht mehr unbedingt nötig, wenn die obere grenze eh der SLAM ist, oder? 
%  The D455 has a depth frame rate of up to 90, while the T265 only achieves a maximum frame rate of 30
The recorded raw data is not directly sent to the plane detection algorithm but instead given to RTAB-MAP, which then performs
calculations to update and publish the map.
Therefore, the upper limit is the frequency of how often RTAB-MAP publishes those updates, which by default is once per second.
According to this upper limit, we consider an algorithm \textit{real-time applicable}, if it achieves an average frame
rate of minimum 1, e.g., the algorithm manages to process the entire point cloud and detect all planes within one second.

%TODO  Eig nur needed wenn die algos langsamer sind als 1s ODER sollte die cloud extrem wachsen kann man sich auf die 6 meter beschränken, erstmal aber nicht}\\
% \subsection*{reduktion (opt)}
% We can reduce complexity further by taking the specifications (background)
% of the D455 into account. The RMS error of the D455 is reported to be 2\% at 4 meters distance to the sensor.
% Furthermore, the ideal distance is stated to range between $0.6 - 6$ meters.
% To maintain a dense and precise representation of our environment, we therefore limit the detection of planes to a
% radius of 6 meters from the current position.

\section{Summary}
Many applications have constraints in the form of a temporal component. Augmented or Virtual Reality applications that include plane detection
are no exception. In addition to time constraints, good quality is usually tightly coupled to expensive sensors.
To evaluate to what extent it is possible to perform precise plane detection with a real-time constraint on off the shelf hardware,
we compare selected algorithms.

\end{document}