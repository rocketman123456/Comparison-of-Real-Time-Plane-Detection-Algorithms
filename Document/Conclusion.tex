\documentclass[main.tex]{subfiles}
\begin{document}
\chapter{Conclusion and Future Work}
\label{chap:concl}

\paragraph{intro summary}
\begin{itemize}
    \item modern man-made environments contain a large number of planar structures, especially indoors
    \item automatic plane detection has become a vital part of many applications and systems   
    \item if a time constraint for the calculations exist, it is often referred to as real-time 
    \item real-time plane detection does exist. However, the sensors that enable real-time plane detection are usually not affordable for the general user
    \item In this thesis, we compared a selection of plane detection algorithms to evaluate the extents of real-time plane detection on affordable hardware like intel realsense 
\end{itemize}


\paragraph{concept summary}
\begin{itemize}
    \item A set of algorithms, appropriate datasets, and a definition of real-time are needed to perform this evaluation.
    \item in subsection X we introduced relevant criteria for the selection of algorithms.
    \item Based on these criteria, we filtered the current state of the art and stuck with four algorithms, namely XY
    \item We proceeded to select suitable datasets for our evaluation, starting with a set of requirements. 
    \item We selected the 2D-3D-S dataset and, because our research did not yield a sufficient match, we created the FIN dataset
    \item Closing Chapter 3, we gave two definitions of realtime
\end{itemize}
\paragraph{evaluation summary}
\paragraph{results summary}
\paragraph{conclusion}
\subsection{Limitations}
\subsection{Future Work}


\paragraph{Use case, scenario, real-world application, current problem}
Viele AR/VR anwendungen benutzen plane detection..
Echtzeit ist wichtig..
Es werden spezielle Sensoren benötigt, ein SLAM algo und ein geeigneter plane detection algo..
Die Auswahl des besten plane detection algorithmus ist nicht trivial..

\paragraph{Daher das thema dieser arbeit}
Wir vergleichen algorithmen mit fokus auf die echtzeitfähigkeit
wir haben 2 definitionen für echtzeit angegeben..
Wir beschränken uns in dieser arbeit auf plane detection mit intel realsense technologie
 das beinhaltet 2 kameras und die dazugehörige software (+SLAM)

\paragraph{algorithmen}
Es wurde die aktuelle literatur gelesen und eine liste aus algorithmen erstellt.
Es wurden metriken zum objektiven und oberflächlichen Vergleich der algorithmen angegeben
basierend auf den metriken wurden vier algorithmen für einen vergleich ausgewählt

\paragraph{Testscenario / datensatz}
Da die ausgewählten algorithmen nicht auf dem selben Datensatz getestet wurden haben wir 
geeignete ausgesucht.
Wir haben uns für den \textit{2D-3D-S} Datensatz entschieden..
motiviert durch die abwesenheit eines datensatzes mit temporaler komponente
 haben wir einen eigenen datensatz erstellt

\paragraph{Experimentaufbau}
Da wir zwei datensätze haben haben wir auch zwei experimente
Zunächst wurden bewertungsmetriken der datensätze eingeführt:
precision, recall, f1, 
dazu wurden die berechnungszeiten genau gemessen und es wird unterschieden in pre-processing,
plane detection und post-processing. 
diese experimente wurden seperat durchgeführt und ausgewertet
im anschluss darauf wurden die ergebnisse beider Experimente verglichen

\paragraph{Ergebnisse der Experimente }
die ergebnisse lassen sich zusammenfassen in:
3D-KHT ist der insgesamt schnellste
RSPD ist der insgesamt präziseste
 
wir haben zwei definitionen von echtzeit
gemäß dieser definitionen und anhand der ergebnisse aus dem letzten kapitel gilt:
(in einem realwelt environment)
\begin{itemize}
    \item 3D-KHT $\in RT_{tot}$, da $t_{tot} < 1$
    \item RSPD $\in RT_{calc}$ , da $t_{calc}< 1$
\end{itemize}

\subsection{Fazit}

OBRG ist nicht echtzeitfähig.

Da das pre-processing von RSPD viel zu lange dauert ist RSPD insgesamt nicht total echtzeitfähig,
die echtzeitfähigkeit beschränkt sich somit auf die plane detection phase.

OPS erzielt im FIN experiment so grade werte die für eine plane detection echtzeit sprechen würden.
Auch hier dauert das pre-processing zu lange um von einer totalen echtzeit sprechen zu können.

3D-KHT ist im FIN experiment deutlich unter der Schranke von einer sekunde, damit in einem realen 
enviroment total echtzeitfähig. 


\paragraph{Limitationen der ergebnisse}
Ein Faktor der langsamen laufzeit von OBRG ist definitiv die implementation. Python ist aus diversen
Gründen magnituden langsamer als C++ und es wurde bei der anfertigung der implementation die 
präzision priorosiert da durch die wahl der programmiersprache sowieso nicht von echtzeitfähigkeit 
ausgegangen wurde.

3D-KHT ist der schnellste algorithmus, lässt aber leider in sachen accuracy viele prozente liegen.

\paragraph{Algo x ist der beste, mögl verbesserungen}



Das einzige was RPSD von einer eindeutigen dominanz trennt ist die Dauer des preprocessings.
In RPSD's pre-processing phase werden die normalen der Punktwolke berechnet. 
Es ist möglich dass die normalen noch vor der plane detection phase aus Figure~\ref{fig:concept}
berechnet werden. Zb kann sogar die intel realsense technologie die normalen exportieren.
Dies würde nach angemessenen anpassungen des algorithmuses die Berechnungszeiten auf $t_{calc}$
reduzieren. Somit würde RSPD ebenfalls total echtzeitfähig sein. 


Diese anpassung des experiments würde einen bias gegenüber RSPD darstellen, weswegen wir das 
unterlassen haben und als weitere aufgabe nach dem abschluss dieser arbeit ansehen.

Ebenso interessant sind die Laufzeiten einer optimierten implementierung von OBRG in C++.

Die parametrisierung von 3D-KHT ist definitiv die komplizierteste aus den algorithmen und hat daher
das größte optimierungspotential. Das ist in dem fall ein doppelschneidiges schwert, da zuerst eine
optimale/ausreichende parametrisierung gefunden werden muss, bevor der algorithmus das volle potential 
entfaltet. 

\end{document}